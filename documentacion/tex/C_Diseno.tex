\apendice{Especificación de diseño}

\section{Introducción}

En esta sección se expone cómo se han resuelto las especificaciones indicadas anteriormente.

Se definen los datos que va a utilizar la aplicación, el diseño de sus interfaces y su arquitectura.

\section{Diseño de datos}

\subsection{Base de datos}

La aplicación utiliza una base de datos SQLite para almacenar los proyectos. La estructura de la base de datos se compone de dos tablas:

\begin{itemize}
\item
\textbf{projects}: En esta tabla se almacenan los nombres de los proyectos. Dispone de un único campo llamado \emph{name}, el cual es clave primaria de la tabla.
\item
\textbf{data}: Esta tabla almacena los datos identificados en cada proyecto. Dispone de 4 campos:
\begin{itemize}
\item
\textbf{name}: Almacena la etiqueta con la que se ha identificado al dato.
\item
\textbf{ID}: Almacena la ID del \emph{frame} que contiene el dato identificado.
\item
\textbf{byte}: Almacena el \emph{byte} del frame que contiene el dato identificado.
\item
\textbf{projectName}: Clave foránea a la tabla \emph{projects}. Nos indica el proyecto al que pertenece el dato identificado.
\end{itemize}
\end{itemize}

A continuación se expone un diagrama relacional de la base de datos:

\imagen{databaseDiagram}{Diagrama relacional de la base de datos}
\newpage
\section{Diseño procedimental}

\imagen{processDiagram}{Diagrama de secuencia de la aplicación.}

Como podemos observar en el diagrama de secuencia, desde el puto de inicio del programa, podemos seleccionar tres ramas:
\begin{itemize}
\item
Comenzar un análisis
\item
Comenzar una monitorización.
\item
Comenzar una lectura de los datos en formato \emph{bruto}.
\end{itemize}

Para llegar a todas las opciones anteriores, es necesario haber configurado la interfaz con la que nos vamos a conectar al bus CAN.

Tanto la primera como la segunda línea del procedimiento hacen uso de la base de datos y de la gestión de proyectos, sin embargo la última rama no hace uso de la base de datos, ya que simplemente se encarga de monitorizar todos los datos en tiempo real en una tabla, sin realizar un almacenaje en disco.

\section{Diseño arquitectónico}

\subsection{Model-View-Controller (MVC)}

Aplicando este patrón de diseño, se utilizan 3 componentes, las vistas, los modelos y los controladores, de tal forma que se separa la lógica de la vista de la aplicación. Con el uso de este método, si realizamos una modificación en una parte de nuestro código, la otra parte no se ve afectada.

Por ejemplo, si modificamos la base de datos, solo modificaríamos el modelo encargado de esa acción, sin tener que modificar por ejemplo, la parte de la vista.

Está compuesto por tres componentes:

\textbf{Model:} Normalmente esta parte se encarga de los datos (no siempre). Esto puede ser por ejemplo, consultando a una base de datos para obtener información.

\textbf{View:} Componen la representación visual de los datos, es decir, todo lo que tenga que ver con la interfaz gráfica.

\textbf{Controller:} Es un mediador entre los modelos y las vistas. Se encarga de recibir las órdenes del usuario a través de la vista, realizar la petición de datos al modelo y devolverlo de nuevo a la vista para que sea mostrado al usuario.

\imagen{modeloMVC}{Modelo MVC.}
\newpage
\subsection{Arquitectura general}

A continuación se describe de manera general la arquitectura de la aplicación. Como se ha indicado anteriormente, se utiliza el modelo MVC para el funcionamiento interno de la aplicación.

Además de eso, cabe destacar la interacción con la base de datos, tanto de lectura y escritura, como la interacción con la interfaz CAN encargada de leer los datos que fluyen por el bus.

\imagen{arquitectura}{Arquitectura general del proyecto.}

\subsection{Diseño de paquetes}

Para conseguir una organización interna del proyecto, se agruparon las distintas funcionalidades de la aplicación en paquetes. De esta manera se consigue una aplicación modular y con funcionalidades independientes.

\imagen{paquetes}{Paquetes de la aplicación}

A continuación se describen cada uno de los paquetes:

\begin{itemize}
\item
\textbf{can}: En este paquete se almacena todo lo relacionado con la interacción con las interfaces CAN.
\item
\textbf{db}: En este paquete se almacena todo lo relacionado con la conexión a la base de datos, la obtención y la inserción de datos en la misma.
\item
\textbf{gui}: En este paquete se almacena todo lo relacionado con la interfaz gráfica. 
\item
\textbf{main}: En este paquete se encuentra el \emph{main} de la aplicación, así como el \emph{preloader}.
\item
\textbf{projects}: En ete paquete se almacena todo lo relacionado con la gestión de proyectos de la aplicación.
\item
\textbf{staticData}: En este paquete se almacenan una serie de estructuras las cuales son utilizadas por distintos componentes del proyecto. En el se almacenan todos los datos que son representados por las gráficas de la aplicación.
\end{itemize}

\imagen{paquetesClases}{Diagrama de clases y paquetes de la aplicación.}

\imagen{paquetesClasesGui}{Diagrama de clases y paquetes de la interfaz gráfica.}



\subsection{Diseño de clases}

En este apartado se describe cada una de las clases que forman la aplicación y su funcionamiento.

\begin{landscape}
\imagenflotante{CRET}{Diagrama UML de la aplicación.}
\end{landscape}

\newpage
\section{Diseño de interfaces}

El diseño de las interfaces se ha llevado a cabo a través de la utilidad SceneBuilder. Se ha intentado seguir una estructura limpia y clara, para facilitar su interacción con el usuario.

A continuación se muestran algunas de las vistas de la aplicación:

\imagen{preloader}{Preloader de la aplicación.}

\imagen{analysis1}{Vista de análisis de la aplicación.}

\imagen{realTimeTag}{Vista de la monitorización de la aplicación.}

El siguiente diagrama nos muestra el flujo de navegabilidad por la aplicación.


\imagen{navegacion}{Diagrama de navegabilidad}.


