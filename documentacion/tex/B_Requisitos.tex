\apendice{Especificación de Requisitos}

\section{Introducción}

A continuación se realiza la especificación de requisitos que define el comportamiento del sistema desarrollado en el proyecto.


\section{Objetivos generales}

Los objetivos generales del proyecto son los siguientes:

\begin{itemize}
\item
Desarrollar una aplicación para el análisis y monitorización de los datos que fluyen por el bus CAN.
\item
Desarrollo de un \emph{hardware} libre el cual permita conectarse y monitorizar dos buses de datos de forma simultánea.
\end{itemize}

\section{Catalogo de requisitos}

En este apartado se describen los requisitos específicos del proyecto, divididos en funcionales y no funcionales.

\subsection{Requisitos funcionales}


\begin{itemize}
\item
\textbf{RF-1: Gestor de proyectos:} La aplicación debe de ser capaz de gestionar proyectos.
\begin{itemize}
\item
\textbf{RF-1.1: Crear proyecto:} El usuario debe de poder crear un nuevo proyecto con un nombre específico.
\begin{itemize}
\item
\textbf{RF-1.1.1: Añadir datos al proyecto:} El usuario debe de poder añadir los datos identificados al proyecto.
\end{itemize}
\item
\textbf{RF-1.2: Listar proyectos:} El usuario debe de poder listar los proyectos existentes en la aplicación.
\item
\textbf{RF-1.3: Eliminar proyectos:} El usuario debe de poder eliminar los proyectos existentes en la aplicación.
\item
\textbf{RF-1.3.1: Confirmar la eliminación:} El usuario debe de poder confirmar si desea eliminar el proyecto.
\item
\textbf{RF-1.4: Abrir proyecto:} El usuario debe de poder abrir un proyecto existente en la aplicación.
\item
\textbf{RF-1.5: Cerrar proyecto:} El usuario debe de poder cerrar un proyecto que esté abierto.
\item
\textbf{RF-1.6: Importar proyecto:} El usuario debe ser capaz de importar un proyecto a la aplicación.
\item
\textbf{RF-1.7: Exportar proyecto:} El usuario debe ser capaz de exportar un proyecto de la aplicación.
\end{itemize}
\item
\textbf{RF-2: Configuración de las interfaces:} El usuario debe de poder configurar las interfaces CAN correspondientes.
\begin{itemize}
\item
\textbf{RF-2.1: Obtención de los puertos serie:} El usuario debe de poder obtener un listado de los puertos serie disponibles en el equipo que ejecuta la aplicación.
\item
\textbf{RF-2.2: Configuración de velocidad:} El usuario debe de ser capaz de configurar la velocidad con la que una interfaz va a ser conectada.
\item
\textbf{RF-2.3: Configuración del modo:} El usuario debe de ser capaz de configurar el modo con el que quiere conectarse al bus CAN.
\end{itemize}
\item
\textbf{RF-3: Configuración del análisis:} El usuario debe de ser capaz de configurar el tipo de análisis que quiere realizar.
\begin{itemize}
\item
\textbf{RF-3.1: Ignorar \emph{bytes} con valor cero:} El usuario debe ser capaz de seleccionar si desea ignorar los \emph{bytes} que no envíen ningún valor por el bus.
\item
\textbf{RF-3.2: Separar los \emph{bytes} en dos partes:} El usuario debe de ser capaz de separar los \emph{bytes} en dos partes para mejorar el análisis si fuera necesario.
\end{itemize}
\item
\textbf{RF-4: Etiquetado de datos:} El usuario debe de ser capaz de etiquetar los datos una vez identificados.
\item
\begin{itemize}
\textbf{RF-4.1: Monitorización de los datos:} El usuario debe de ser capaz de monitorizar los datos una vez han sido etiquetados.
\end{itemize}
\end{itemize}


\subsection{Requisitos no funcionales}

\begin{itemize}
\item
\textbf{RNF-1: Monitorización:} La aplicación debe monitorizar correctamente los datos que están siendo analizados siempre que se esté conectado al bus.
\item
\textbf{RNF-2: Rendimiento:} La aplicación debe funcionar fluidamente, sin que la interfaz gráfica se quede bloqueada.
\item
\textbf{RNF-3: Escalabilidad:} La aplicación debe permitir la incorporación de módulos de forma sencilla.
\item
\textbf{RNF-4: Usabilidad:} La aplicación debe ser fácil de utilizar, intuitiva y con una estructura clara.
\end{itemize}
\newpage
\section{Especificación de requisitos}

En esta sección se exponen los casos de uso de la aplicación.

\subsection{Actores}

El único actor del sistema será la persona que maneja la aplicación y que identificará las señales importantes.

\subsection{Casos de uso}
\imagen{umlCasosUso}{Diagrama UML general de los casos de uso.}

\tablaSmallSinColores{Caso de uso 1: Gestión de proyectos }{p{3cm} p{.75cm} p{9.5cm}}{b1}{
\multicolumn{3}{l}{Caso de uso 1: Gestión de proyectos} \\
}
{
Descripción & \multicolumn{2}{p{10.25cm}}{Permite gestionar los proyectos de la aplicación.} \\\hline
\multirow{2}{3.5cm}{Requisitos} &\multicolumn{2}{p{10.25cm}}{RF-1} \\\cline{2-3}
& \multicolumn{2}{p{10.25cm}}{RF-1}
\\\cline{2-3}
& \multicolumn{2}{p{10.25cm}}{RF-1.1}
\\\cline{2-3}
& \multicolumn{2}{p{10.25cm}}{RF-1.2}
\\\cline{2-3}
& \multicolumn{2}{p{10.25cm}}{RF-1.3}
\\\cline{2-3}
& \multicolumn{2}{p{10.25cm}}{RF-1.4}
\\\cline{2-3}
& \multicolumn{2}{p{10.25cm}}{RF-1.5}
\\\cline{2-3}
& \multicolumn{2}{p{10.25cm}}{RF-1.6}
\\\cline{2-3}
& \multicolumn{2}{p{10.25cm}}{RF-1.7}
\\\hline
Precondiciones & \multicolumn{2}{p{10.25cm}} {Existe una conexión a la base de datos}
\\\hline
\multirow{2}{3.5cm}{Secuencia normal} & Paso & Acción \\\cline{2-3}
& 1 & El usuario ejecuta la aplicación.
\\\cline{2-3}
& 2 & Se listan los proyectos disponibles.
\\\cline{2-3}
& 3 & Por cada proyecto, existe la opción de abrir y eliminar.
\\\cline{2-3}
& 4 & Es posible importar un poryecto.
\\\cline{2-3}
& 5 & Es posible exportar un poryecto.
\\\hline
Postcondiciones & \multicolumn{2}{p{10.25cm}}{Se muestran todos los proyectos.} \\\hline
Excepciones & \multicolumn{2}{p{10.25cm}}{Error en la conexión con la base de datos.}\\\hline
Importancia & Alta \\\hline
Urgencia & Alta \\\hline
Comentarios & & \\
}


\tablaSmallSinColores{Caso de uso 2: Listar proyectos }{p{3cm} p{.75cm} p{9.5cm}}{b1}{
\multicolumn{3}{l}{Caso de uso 2: Listar proyectos} \\
}
{
Descripción & \multicolumn{2}{p{10.25cm}}{Permite listar los proyectos de la aplicación.} \\\hline
\multirow{2}{3.5cm}{Requisitos} &\multicolumn{2}{p{10.25cm}}{RF-1.2} \\
\\\hline
Precondiciones & \multicolumn{2}{p{10.25cm}} {Existe una conexión a la base de datos}
\\\hline
\multirow{2}{3.5cm}{Secuencia normal} & Paso & Acción \\\cline{2-3}
& 1 & El usuario ejecuta la aplicación.
\\\cline{2-3}
& 2 & El usuario lista los proyectos disponibles.
\\\hline
Postcondiciones & \multicolumn{2}{p{10.25cm}}{Se muestran todos los proyectos.} \\\hline
Excepciones & \multicolumn{2}{p{10.25cm}}{Error en la conexión con la base de datos.}\\\hline
Importancia & Alta \\\hline
Urgencia & Alta \\\hline
Comentarios & & \\
}


\tablaSmallSinColores{Caso de uso 3: Crear proyecto }{p{3cm} p{.75cm} p{9.5cm}}{b1}{
\multicolumn{3}{l}{Caso de uso 3: Crear proyecto} \\
}
{
Descripción & \multicolumn{2}{p{10.25cm}}{Permite crear un nuevo proyecto y añadirle datos.} \\\hline
\multirow{2}{3.5cm}{Requisitos}&\multicolumn{2}{p{10.25cm}}{RF-1.1}
\\\cline{2-3}
& \multicolumn{2}{p{10.25cm}}{RF-1.2}
\\\hline
Precondiciones & \multicolumn{2}{p{10.25cm}} {Existe una conexión a la base de datos}
\\\hline
\multirow{2}{3.5cm}{Secuencia normal} & Paso & Acción \\\cline{2-3}
& 1 & El usuario ejecuta la aplicación.
\\\cline{2-3}
& 2 & Se crea un nuevo proyecto.
\\\cline{2-3}
& 3 & Se guardan los datos del proyecto.
\\\hline
Postcondiciones & \multicolumn{2}{p{10.25cm}}{Se muestran todos los proyectos.} \\\hline
Excepciones & \multicolumn{2}{p{10.25cm}}{Error en la conexión con la base de datos.}\\\hline
Importancia & Alta \\\hline
Urgencia & Alta \\\hline
Comentarios & & \\
}


\tablaSmallSinColores{Caso de uso 4: Eliminar proyecto}{p{3cm} p{.75cm} p{9.5cm}}{b1}{
\multicolumn{3}{l}{Caso de uso 4: Eliminar proyecto} \\
}
{
Descripción & \multicolumn{2}{p{10.25cm}}{Permite eliminar un proyecto existente.} \\\hline
\multirow{2}{3.5cm}{Requisitos}&\multicolumn{2}{p{10.25cm}}{RF-1.3}
\\\cline{2-3}
& \multicolumn{2}{p{10.25cm}}{RF-1.3.1}
\\\hline
Precondiciones & \multicolumn{2}{p{10.25cm}} {Existe una conexión a la base de datos}
\\\hline
\multirow{2}{3.5cm}{Secuencia normal} & Paso & Acción \\\cline{2-3}
& 1 & El usuario ejecuta la aplicación.
\\\cline{2-3}
& 2 & Se crea un nuevo proyecto.
\\\cline{2-3}
& 3 & Se guardan los datos del proyecto.
\\\hline
Postcondiciones & \multicolumn{2}{p{10.25cm}}{Se elimina el proyecto seleccionado.} \\\hline
Excepciones & \multicolumn{2}{p{10.25cm}}{El proyecto no existe.}\\\hline
Importancia & Alta \\\hline
Urgencia & Alta \\\hline
Comentarios & & \\
}

\tablaSmallSinColores{Caso de uso 5: Abrir proyecto}{p{3cm} p{.75cm} p{9.5cm}}{b1}{
\multicolumn{3}{l}{Caso de uso 5: Abrir proyecto} \\
}
{
Descripción & \multicolumn{2}{p{10.25cm}}{Permite abrir un proyecto existente.} \\\hline
\multirow{2}{3.5cm}{Requisitos} &\multicolumn{2}{p{10.25cm}}{RF-1.4} \\
\\\hline
Precondiciones & \multicolumn{2}{p{10.25cm}} {Existe una conexión a la base de datos}
\\\hline
\multirow{2}{3.5cm}{Secuencia normal} & Paso & Acción \\\cline{2-3}
& 1 & El usuario ejecuta la aplicación.
\\\cline{2-3}
& 2 & El usuario lista los proyectos disponibles.
\\\cline{2-3}
& 3 & El usuario selecciona el proyecto que desea abrir.
\\\cline{2-3}
& 4 & El usuario abre el proyecto.
\\\hline
Postcondiciones & \multicolumn{2}{p{10.25cm}}{Se abre el proyecto seleccionado.} \\\hline
Excepciones & \multicolumn{2}{p{10.25cm}}{El proyecto no existe.}\\\hline
Importancia & Alta \\\hline
Urgencia & Alta \\\hline
Comentarios & & \\
}


\tablaSmallSinColores{Caso de uso 6: Cerrar proyecto}{p{3cm} p{.75cm} p{9.5cm}}{b1}{
\multicolumn{3}{l}{Caso de uso 6: Cerrar proyecto} \\
}
{
Descripción & \multicolumn{2}{p{10.25cm}}{Permite cerrar un proyecto abierto.} \\\hline
\multirow{2}{3.5cm}{Requisitos} &\multicolumn{2}{p{10.25cm}}{RF-1.5} \\
\\\hline
Precondiciones & \multicolumn{2}{p{10.25cm}} {Existe un proyecto abierto.}
\\\hline
\multirow{2}{3.5cm}{Secuencia normal} & Paso & Acción \\\cline{2-3}
& 1 & El usuario ejecuta la aplicación.
\\\cline{2-3}
& 2 & El usuario tiene un proyecto abierto en la aplicación.
\\\cline{2-3}
& 3 & El usuario cierra el proyecto.
\\\cline{2-3}
& 4 & El usuario confirma que desea guardar los cambios.
\\\cline{2-3}
& 5 & Se cierra el proyecto.
\\\hline
Postcondiciones & \multicolumn{2}{p{10.25cm}}{Se cierra el proyecto abierto.} \\\hline
Excepciones & \multicolumn{2}{p{10.25cm}}{No hay conexión con la base de datos.}\\\hline
Importancia & Alta \\\hline
Urgencia & Alta \\\hline
Comentarios & & \\
}


\tablaSmallSinColores{Caso de uso 7: Importar proyecto}{p{3cm} p{.75cm} p{9.5cm}}{b1}{
\multicolumn{3}{l}{Caso de uso 7: Importar proyecto} \\
}
{
Descripción & \multicolumn{2}{p{10.25cm}}{Permite importar un proyecto.} \\\hline
\multirow{2}{3.5cm}{Requisitos} &\multicolumn{2}{p{10.25cm}}{RF-1.6} \\
\\\hline
Precondiciones & \multicolumn{2}{p{10.25cm}} {Existe conexión a la base de datos.}
\\\hline
\multirow{2}{3.5cm}{Secuencia normal} & Paso & Acción \\\cline{2-3}
& 1 & El usuario ejecuta la aplicación.
\\\cline{2-3}
& 2 & El usuario lista los proyectos disponibles.
\\\cline{2-3}
& 3 & El usuario selecciona el fichero que desea importar
\\\cline{2-3}
& 4 & El usuario lista los proyectos disponibles de nuevo.
\\\cline{2-3}
& 5 & El proyecto importado se encuentra en la aplicación.
\\\hline
Postcondiciones & \multicolumn{2}{p{10.25cm}}{Existe un nuevo proyecto en la aplicación.} \\\hline
Excepciones & \multicolumn{2}{p{10.25cm}}{No hay conexión con la base de datos.}\\\hline
Importancia & Alta \\\hline
Urgencia & Alta \\\hline
Comentarios & & \\
}

\tablaSmallSinColores{Caso de uso 8: Exportar proyecto}{p{3cm} p{.75cm} p{9.5cm}}{b1}{
\multicolumn{3}{l}{Caso de uso 8: Exportar proyecto} \\
}
{
Descripción & \multicolumn{2}{p{10.25cm}}{Permite exportar un proyecto.} \\\hline
\multirow{2}{3.5cm}{Requisitos} &\multicolumn{2}{p{10.25cm}}{RF-1.7} \\
\\\hline
Precondiciones & \multicolumn{2}{p{10.25cm}} {Existe conexión a la base de datos.}
\\\hline
\multirow{2}{3.5cm}{Secuencia normal} & Paso & Acción \\\cline{2-3}
& 1 & El usuario ejecuta la aplicación.
\\\cline{2-3}
& 2 & El usuario lista los proyectos disponibles.
\\\cline{2-3}
& 3 & El usuario selecciona el proyecto que desea exportar.
\\\cline{2-3}
& 4 & El usuario selecciona dónde desea guardar el proyecto.
\\\cline{2-3}
& 5 & El proyecto es exportado a un fichero.
\\\hline
Postcondiciones & \multicolumn{2}{p{10.25cm}}{Se exporta un proyecto de la aplicación.} \\\hline
Excepciones & \multicolumn{2}{p{10.25cm}}{No existe ningún proyecto}\\\hline
Importancia & Alta \\\hline
Urgencia & Alta \\\hline
Comentarios & & \\
}

\tablaSmallSinColores{Caso de uso 9: Obtención de puertos serie disponibles}{p{3cm} p{.75cm} p{9.5cm}}{b1}{
\multicolumn{3}{l}{Caso de uso 9: Obtención de puertos serie disponibles} \\
}
{
Descripción & \multicolumn{2}{p{10.25cm}}{Permite obtener los puertos series disponibles en el equipo.} \\\hline
\multirow{2}{3.5cm}{Requisitos}&\multicolumn{2}{p{10.25cm}}{RF-2}
\\\cline{2-3}
& \multicolumn{2}{p{10.25cm}}{RF-2.1}
\\\hline
Precondiciones & \multicolumn{2}{p{10.25cm}} {Existe conexión a la base de datos.}
\\\hline
\multirow{2}{3.5cm}{Secuencia normal} & Paso & Acción \\\cline{2-3}
& 1 & El usuario ejecuta la aplicación.
\\\cline{2-3}
& 2 & El usuario se dispone a configurar una de las interfaces.
\\\cline{2-3}
& 3 & El usuario lista los puertos serie disponibles.
\\\hline
Postcondiciones & \multicolumn{2}{p{10.25cm}}{Se listan los puertos serie.} \\\hline
Excepciones & \multicolumn{2}{p{10.25cm}}{No existe ningún puerto serie disponible}\\\hline
Importancia & Alta \\\hline
Urgencia & Alta \\\hline
Comentarios & & \\
}


\tablaSmallSinColores{Caso de uso 10: Configuración de la interfaz CAN}{p{3cm} p{.75cm} p{9.5cm}}{b1}{
\multicolumn{3}{l}{Caso de uso 10: Configuración de la interfaz CAN} \\
}
{
Descripción & \multicolumn{2}{p{10.25cm}}{Permite configurar y conectarse a una interfaz CAN.} \\\hline
\multirow{2}{3.5cm}{Requisitos}&\multicolumn{2}{p{10.25cm}}{RF-2.1}
\\\cline{2-3}
& \multicolumn{2}{p{10.25cm}}{RF-2.2}
\\\cline{2-3}
& \multicolumn{2}{p{10.25cm}}{RF-2.3}
\\\cline{2-3}
& \multicolumn{2}{p{10.25cm}}{RF-3}
\\\cline{2-3}
& \multicolumn{2}{p{10.25cm}}{RF-3.1}
\\\cline{2-3}
& \multicolumn{2}{p{10.25cm}}{RF-3.2}
\\\hline
Precondiciones & \multicolumn{2}{p{10.25cm}} {Existe un puerto serie en la máquina.}
\\\hline
\multirow{2}{3.5cm}{Secuencia normal} & Paso & Acción \\\cline{2-3}
& 1 & El usuario ejecuta la aplicación.
\\\cline{2-3}
& 2 & El usuario selecciona un puerto serie de la máquina.
\\\cline{2-3}
& 4 & El usuario configura la velocidad de conexión.
\\\cline{2-3}
& 5 & El usuario configura el modo de conexión.
\\\cline{2-3}
& 6 & El usuario configura si desea ignorar los \emph{bytes} con valor cero.
\\\cline{2-3}
& 7 & El usuario configura si desea dividir los \emph{bytes} en dos partes.
\\\hline
Postcondiciones & \multicolumn{2}{p{10.25cm}}{Se conecta a una interfaz CAN.} \\\hline
Excepciones & \multicolumn{2}{p{10.25cm}}{El puerto serie está en uso}\\\hline
Importancia & Alta \\\hline
Urgencia & Alta \\\hline
Comentarios & & \\
}


\tablaSmallSinColores{Caso de uso 11: Etiquetado de datos}{p{3cm} p{.75cm} p{9.5cm}}{b1}{
\multicolumn{3}{l}{Caso de uso 11: Etiquetado de datos} \\
}
{
Descripción & \multicolumn{2}{p{10.25cm}}{Permite asignar etiquetas a los datos identificados por el usuario.} \\\hline
\multirow{2}{3.5cm}{Requisitos} &\multicolumn{2}{p{10.25cm}}{RF-4} \\
\\\hline
Precondiciones & \multicolumn{2}{p{10.25cm}} {Existe un dato en el programa.}
\\\hline
\multirow{2}{3.5cm}{Secuencia normal} & Paso & Acción \\\cline{2-3}
& 1 & El usuario ejecuta la aplicación.
\\\cline{2-3}
& 2 & El usuario configura una interfaz.
\\\cline{2-3}
& 4 & El usuario crea un nuevo proyecto.
\\\cline{2-3}
& 5 & El usuario comienza el análisis.
\\\cline{2-3}
& 6 & El usuario identifica uno de los datos.
\\\cline{2-3}
& 7 & El usuario establece una etiqueta al dato.
\\\cline{2-3}
& 8 & El usuario envía el dato al dashboard.
\\\hline
Postcondiciones & \multicolumn{2}{p{10.25cm}}{Etiqueta un dato identificado.} \\\hline
Excepciones & \multicolumn{2}{p{10.25cm}}{No se ha creado un proyecto.}\\\hline
Importancia & Alta \\\hline
Urgencia & Alta \\\hline
Comentarios & & \\
}


\tablaSmallSinColores{Caso de uso 12: Monitorización de los datos}{p{3cm} p{.75cm} p{9.5cm}}{b1}{
\multicolumn{3}{l}{Caso de uso 12: Monitorización de los datos} \\
}
{
Descripción & \multicolumn{2}{p{10.25cm}}{Permite monitorizar los datos identificados} \\\hline
\multirow{2}{3.5cm}{Requisitos} &\multicolumn{2}{p{10.25cm}}{RF-4.1} \\
\\\hline
Precondiciones & \multicolumn{2}{p{10.25cm}} {Existen datos identificados.}
\\\hline
\multirow{2}{3.5cm}{Secuencia normal} & Paso & Acción \\\cline{2-3}
& 1 & El usuario ejecuta la aplicación.
\\\cline{2-3}
& 2 & El usuario configura una interfaz.
\\\cline{2-3}
& 4 & El usuario lista los proyectos disponibles.
\\\cline{2-3}
& 5 & El usuario abre uno de los proyectos.
\\\cline{2-3}
& 6 & El usuario comienza la monitorización de los datos.
\\\hline
Postcondiciones & \multicolumn{2}{p{10.25cm}}{Se muestran los datos que circulan por el bus.} \\\hline
Excepciones & \multicolumn{2}{p{10.25cm}}{No existe un proyecto..}\\\hline
Importancia & Alta \\\hline
Urgencia & Alta \\\hline
Comentarios & & \\
}

