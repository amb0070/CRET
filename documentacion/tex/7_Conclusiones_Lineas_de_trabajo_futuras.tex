\capitulo{7}{Conclusiones y Líneas de trabajo futuras}

\section{Conclusiones}\label{conclusiones}

Una vez concluido el proyecto, es posible extraer las siguientes conclusiones:

Se ha cumplido el objetivo general del proyecto. Ahora es mucho más fácil realizar un análisis de las señales que fluyen por el bus, así como su monitorización.

Para conseguir el desarrollo del proyecto, ha sido necesaria la adquisición de nuevos conocimientos tanto en el ámbito del software, con el uso de JavaFX y la profundización en el lenguaje Java, así como a nivel de hardware, en el que se han adquirido conocimientos básicos pero suficientes para alcanzar el objetivo.



\section{Líneas de trabajo futuras}\label{lineas_de_trabajo_futuras}

Es interesante proseguir con el desarrollo de la herramienta y adaptarla para su utilización con los módulos de Linux SocketCAN.
De esta manera, se conseguiría deshacer de la dependencia que existe actualmente con el \emph{hardware}, para la utilización de la aplicación. SocketCAN haría de intermediario entre el \emph{hardware} y el \emph{software} a modo de \emph{driver} de todos aquellos dispositivos que sean soportados.

Otra de las ventajas de realizar esa modificación, sería la posibilidad de crear y utilizar interfaces virtuales dentro del equipo, de manera que ni si quiera sería necesario un hardware físico para realizar pruebas.
