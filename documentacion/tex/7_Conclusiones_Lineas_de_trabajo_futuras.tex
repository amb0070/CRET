\capitulo{7}{Conclusiones y Líneas de trabajo futuras}

\section{Conclusiones}\label{conclusiones}

Una vez concluido el proyecto, es posible extraer las siguientes conclusiones:
\begin{itemize}
\item
Se ha cumplido el objetivo general del proyecto. Ahora es mucho más fácil realizar un análisis de las señales que fluyen por el bus, así como su monitorización.

\item
Para conseguir el desarrollo del proyecto, ha sido necesaria la adquisición de nuevos conocimientos tanto en el ámbito del software, con el uso de JavaFX y la profundización en el lenguaje Java, así como a nivel de \emph{hardware}, en el que se han adquirido conocimientos básicos pero suficientes para alcanzar el objetivo.

\item
Se han utilizado gran cantidad de tecnologías durante el desarrollo del proyecto, desde el diseño de elementos con \emph{software} de modelado 3D pasando por el diseño y producción de \emph{hardware} y el desarrollo de la aplicación en Java con el uso de librerías como JavaFX y Medusa para mejorar la interacción con el usuario.

\end{itemize}

\newpage
\section{Líneas de trabajo futuras}\label{lineas_de_trabajo_futuras}

Es interesante proseguir con el desarrollo de la herramienta y adaptarla para su utilización con los módulos de \emph{Linux} \emph{SocketCAN}.
De esta manera, se conseguiría deshacer de la dependencia que existe actualmente con el \emph{hardware}, para la utilización de la aplicación. \emph{SocketCAN} haría de intermediario entre el \emph{hardware} y el \emph{software} a modo de \emph{driver} de todos aquellos dispositivos que sean soportados.

Otra de las ventajas de realizar esa modificación, sería la posibilidad de crear y utilizar interfaces virtuales dentro del equipo, de manera que ni siquiera sería necesario un \emph{hardware} físico para realizar pruebas.
