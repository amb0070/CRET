\apendice{Documentación técnica de programación}

\section{Introducción}

En este anexo se describe la documentación técnica de programación, incluyendo la preparación del entorno de desarrollo, la estructura de la aplicación y las configuraciones necesarias.

\section{Estructura de directorios}

El proyecto tiene la siguiente estructura:

\begin{itemize}
\item
\textbf{/documentacion/:} Documentación del proyecto.
\item
\textbf{/documentacion/img/:} Imágenes utilizadas en la documentación.
\item
\textbf{/documentacion/javadoc/:} Documentación \emph{Javadoc}.
\item
\textbf{/hardware/:} Esquemas del \emph{hardware} realizado en formato para KiCAD.
\item
\textbf{/hardware/gerber/:} Ficheros \emph{gerber} para la producción del hardware.
\item
\textbf{/src/:} Código fuente de la aplicación Java
\item
\textbf{/resources/:} Recursos de la aplicación.

\end{itemize}


\section{Manual del programador}

El siguiente manual tiene como objetivo servir de referencia para futuros programadores. Se explica como montar el entorno de desarrollo, las dependencias necesarias así como su compilación y ejecución.

\subsection{Entorno de desarrollo}

Para el proyecto se necesita tener instaladas las siguientes dependencias:

\begin{itemize}
\item
Java JDK 8
\item
JavaFX
\item
Git
\item
SceneBuilder
\item
Eclipse
\end{itemize}

\subsection{Java JDK 8}

Es uno de los lenguajes más utilizados hoy en día. El uso del lenguaje Java nos permite la compatibilidad de la aplicación en distintas plataformas. Se puede obtener del siguiente enlace:

ENLACE A JAVA JDK8

\subsection{JavaFX}

Se trata de un conjunto de productos y tecnologías para la creación de interfaces gráficas en Java. Para la ejecución de estas aplicaciones solamente es necesaria la instalación de JRE en la máquina.

Se puede obtener del siguiente enlace:
ENLACE A JAVAFX

\subsection{Git}

Para utilizar el repositorio en el que se encuentra la aplicación, es necesario el uso de Git. Este programa nos permitirá clonar el repositorio a nuestra máquina y empezar a trabajar con ello.

Podemos instalarlo en Windows habilitando el WSL (Windows Subsystem for Linux), y ejecutando el comando \emph{apt-get install git} con permisos de \emph{root}.

\imagen{gitClone}{Comando Git sobre WSL.}

\subsection{SceneBuilder}

Se trata de una interfaz gráfica creada para el desarrollo de interfaces gráficas en JavaFX. Sigue el modelo drag and drop, y toda la información es almacenada en un fichero FXML, un formato especial que extiende de XML.

\imagen{sceneBuilder}{SceneBuilder 10.0}

\subsection{Eclipse}

Para el desarrollo del código Java, se ha utilizado la IDE Eclipse. En ella se han importado las librerías correspondientes, e instalado plugins para el soporte de JavaFX. De esta manera, es posible llamar a SceneBuilder desde Eclipse.

\imagen{eclipse}{IDE Eclipse}

\subsection{Obtención del código fuente}

El código fuente de la aplicación está hospedado en un repositorio Git en GitHub. Para obtener una copia de este código, se pueden seguir los siguientes pasos:

Abrir el WSL en Windows.

Posicionarse en el directorio en el que se desea copiar el código. (Si deseamos acceder al sistema de ficheros de Windows desde WSL, la ruta será /mnt/c).

Introducir el siguiente comando:

\emph{git clone https://github.com/amb0070/CRET.git}

Una vez finalizada la descarga del repositorio, se podrá acceder al código fuente y a los recursos del proyecto.

\imagen{gitClone}{Comando Git sobre WSL.}

\subsection{Importar proyecto a Eclipse}

Una vez descargado el proyecto, abrimos Eclipse.

Hacemos \emph{click} en \emph{File} > \emph{Import...}

Seleccionamos \emph{Projects from File System or Archive}, y seleccionamos el directorio donde hemos clonado el repositorio.

De esta manera, queda importado el proyecto a nuestro \emph{workspace} de Eclipse.

CAPTURAS

\subsection{Importar librerías necesarias}

Para importar las librerías necesarias y utilizadas por el proyecto, podemos seguir los siguientes pasos:

Sobre la carpeta del proyecto, en el árbol izquierdo, hacer click derecho.

Seleccionar sub-menú \emph{Build Path} > \emph{Configure Build Path}.

Hacemos \emph{click} en el botón \emph{Add External JARs...} y seleccionamos los ficheros .jar de las librerías.

CAPTURAS


\section{Compilación, instalación y ejecución del proyecto}


\subsection{Compilación}
La compilación del proyecto se realiza desde Eclipse. 

Para exportarlo como un fichero .jar, podemos hacer click derecho sobre el árbol izquierdo del proyecto, y \emph{click} en \emph{Export...}.

Seleccionamos \emph{Java} > \emph{Runnable JAR file}.

Incluimos las librerías en el fichero .jar, y exportamos.

CAPTURAS

\subsection{Ejecución}

No es necesario realizar una instalación de la aplicación. Simplemente tener instalado en la máquina el JRE.

La ejecución de la aplicación no requiere de ningún elemento externo, pero si que es necesario tener el \emph{hardware} conectado a la máquina para su funcionamiento.

De igual manera, el proyecto puede ser ejecutado desde Eclipse, pero para su funcionamiento es necesario el \emph{hardware} que se ha diseñado.


