\capitulo{2}{Objetivos del proyecto}

A continuación se definen los objetivos del proyecto realizado, estructurados en tres secciones:


\section{Objetivos generales}\label{objetivos-generales}

\begin{itemize}

\item
  Desarrollar una aplicación para el análisis y monitorización de  los datos que fluyen por el bus CAN.
\item
  Desarrollo de un hardware libre el cual permita conectarse a dicho bus de datos y monitorizar distintas velocidades de forma simultanea.

  
\end{itemize}

\section{Objetivos técnicos}\label{objetivos-tecnicos}

\begin{itemize}

\item
  Desarrollar un \emph{hardware} propio desde 0 siguiendo la metodología para el diseño del mismo, así como su producción y montaje.
\item
  Desarrollar una aplicación en Java con el uso de la librería JavaFX para la parte correspondiente a la interfaz gráfica.
\item
  Aplicar la arquitectura MVC (\emph{Model-View-Controller}) en el desarrollo de la aplicación.
\item
  Utilizar Zenhub para realizar un seguimiento y gestión del proyecto durante su desarrollo.
\item
  Utilizar Git (en la plataforma GitHub) para realizar un control de versiones de software.
\item
  Utilizar librerías para la recolección de datos del bus CAN.
\end{itemize}

\section{Objetivos personales}\label{objetivos-personales}

\begin{itemize}

\item
  Profundizar en el conocimiento de \emph{hardware} y en el desarrollo del mismo.
\item
  Adquirir conocimiento sobre el funcionamiento del bus CAN en distintos escenarios.
\item
  Profundizar en el conocimiento del análisis de datos y monitorización en tiempo real.

\end{itemize}