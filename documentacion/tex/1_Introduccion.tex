\capitulo{1}{Introducción}

Todos los vehículos que utilizamos en el día a día, maquinaria utilizada en las empresas, el sector náutico o de la aviación, utilizan el bus CAN para la \emph{interconexión} de los componentes electrónicos que hacen funcionar dichas máquinas.

Los estándares del protocolo son públicos de forma que cualquier desarrollador pueda ceñirse a ellos, pero no sucede así con los datos que circulan por el bus, ni la forma en la que lo hacen.

La forma en la que se envían y reciben esos datos, al igual que el propio dato en si, es definido por cada fabricante, y no se suelen hacer públicos.

Lo que se intenta conseguir a través del uso de esta herramienta, es la posibilidad identificar, clasificar y monitorizar los datos que los distintos elementos de la máquina (un vehículo, por ejemplo) utilizan para su funcionamiento interno y para el intercambio de datos como señales o textos.

De esta manera, por ejemplo, si necesitásemos realizar una aplicación para la monitorización de un vehículo y ciertas señales como la velocidad, las revoluciones del motor o el GPS fueran transmitidas a través del bus CAN, no sería necesario introducir nuevos sensores, sino que estos datos serían extraídos del bus CAN, ahorrando costes, facilitando el desarrollo y evitando posibles problemas.


Para realizar ese análisis existen algunos programas y proyectos disponibles, pero que requieren de un \emph{hardware} con un precio elevado. Además, es necesario una gran cantidad de conocimientos previos para el uso de dichas herramientas.

Con este proyecto, se propone realizar dicha clasificación y monitorización de manera rápida e interactiva, representando los datos visualmente, permitiendo etiquetarlos y almacenarlos de manera que puedan ser utilizados posteriormente.

\section{Estructura de la memoria}\label{estructura-de-la-memoria}

A continuación se describe la estructura de la memoria:

\begin{itemize}

\item

\textbf{Introducción}: Descripción breve sobre el proyecto, motivación por la que se ha realizado y soluciones propuestas. Estructura de la memoria y listado de materiales adjuntos proporcionados.

\item

\textbf{Objetivos del proyecto:} Exposición de los objetivos, clasificados en objetivos generales, objetivos técnicos y objetivos personales.

\item

\textbf{Conceptos teóricos:} Conceptos básicos y necesarios para entender el propósito del proyecto así como su desarrollo.

\item

\textbf{Técnicas y herramientas:} Metodologías y herramientas utilizadas durante el desarrollo del proyecto.

\item

\textbf{Trabajos relacionados:} Aplicaciones, proyectos y empresas que ofrecen soluciones en el mismo campo que el estudiado.

\item

\textbf{Conclusiones y líneas de trabajo futuras:} Conclusiones a las que se ha llegado tras la realización del proyecto, así como mejoras  y futuro desarrollo de la aplicación.

\end{itemize}

\section{Materiales adjuntos}\label{materiales-adjuntos}

Los materiales adjuntos a la memoria son los siguientes:

\begin{itemize}

\item
	Aplicación desarrollada en Java: CRET.
\item	
	Fotos del \emph{hardware} desarrollado.
\item
	Esquemas del \emph{hardware} desarrollado.
\item
	Diseño de la caja preparada para la impresión 3D.
\item	
	Documentación del código fuente, en formato JavaDoc.
\item
	Vídeos de prueba del proyecto.
\end{itemize}

Además, los siguientes recursos están accesibles a través de internet:

\begin{itemize}
\item
  Repositorio del proyecto: \url{https://github.com/amb0070/CRET/}
\end{itemize}
