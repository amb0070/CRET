\capitulo{1}{Introducción}

Todos los vehículos que utilizamos en el día a día, maquinaria utilizada en las empresas, el sector náutico o de la aviación utilizan el bus CAN para la inter-conexión de los componentes electrónicos que hacen funcionar dichas máquinas.

Durante los años se han desarrollado nuevos protocolos, todos ellos basados en  (habiéndose incrementado, por ejemplo, la velocidad en los buses actuales), pero utilizando las bases del protocolo que en su primer desarrollo.

El estándar CAN (CITA) es 

El estándar del bus CAN únicamente hace referencia a las dos primeras capas del protocolo, la capa física, y la capa de enlace de datos (siguiendo el modelo OSI).


A través del uso de esta herramienta, sería posible identificar y clasificar los datos que los distintos elementos del vehículo analizado comparten entre ellos, para su funcionamiento. De esta manera, por ejemplo, si necesitásemos realizar una aplicación para la monitorización de un vehículo, no sería necesario introducir nuevos sensores (para la velocidad, las revoluciones del motor, el GPS), sino que estos datos serían extraídos del bus CAN, ahorrando costes y posibles problemas.

\section{Estructura de la memoria}\label{estructura-de-la-memoria}



Introducción: Descripción breve sobre el proyecto, motivación por la que se ha realizado y soluciones propuestas. Estructura de la memoria y listado de materiales adjuntos proporcionados.

Objetivos del proyecto: Exposición de los objetivos, clasificados en objetivos generales, objetivos técnicos y objetivos personales.

Conceptos teóricos: Conceptos básicos y necesarios para entender el propósito del proyecto. 

Técnicas y herramientas: Metodologías y herramientas utilizadas durante el desarrollo del proyecto.

Trabajos relacionados: Aplicaciones, proyectos y empresas que ofrecen soluciones en el mismo campo que el estudiado.

Conclusiones y líneas de trabajo futuras: Conclusiones a las que se ha llegado tras la realización del proyecto, así como mejoras  y futuro desarrollo de la aplicación.

\section{Materiales adjuntos}\label{materiales-adjuntos}

Los materiales adjuntos a la memoria son los siguientes:

\begin{itemize}
\tightlist
\item
	Aplicación desarrollada en Java: CRET.
\item	
	Fotos del hardware desarrollado.
\item
	Esquemas del hardware desarrollado.
\item	
	JavaDoc.
\end{itemize}

Además, los siguientes recursos están accesibles a través de internet:

\begin{itemize}
\tightlist
\item
  Repositorio del proyecto TODO----.
\end{itemize}
