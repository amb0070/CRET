\apendice{Plan de Proyecto Software}

\section{Introducción}

Una de las fases más importantes de cualquier proyecto es la planificación. En esta fase se estima el tiempo y los requisitos necesarios para la realización del proyecto. Para ello es necesaria una idea clara de lo que se quiere realizar de manera que todas las partes que componen el proyecto puedan ser analizadas individualmente.

Podemos dividir este anexo en dos apartados más pequeños:

\begin{itemize}
\item
  Planificación temporal.
\item
  Estudio de viabilidad.
\end{itemize}

En la primera fase se realiza una planificación de los tiempos esperados. Se especifican los tiempos necesarios para cada una de las partes que componen el proyecto, a la vez que una fecha de inicio y de fin de las mismas.

La segunda fase se centra en un estudio de la viabilidad del proyecto. En esta fase se pueden diferenciar dos partes:

\begin{itemize}
\item
\textbf{Viabilidad económica:} Estimación de los costes y beneficios del proyecto.
\item
\textbf{Viabilidad legal:} Regulaciones legales que pueden afectar al proyecto. En este caso serían la Ley de Protección de Datos y las licencias del software.
\end{itemize}

\section{Planificación temporal}

Para realizar una planificación del proyecto adecuada, se decidió aplicar \emph{Scrum} (una metodología ágil de desarrollo, explicada en la memoria).

Para ello, se dividió el trabajo en \emph{sprints} los cuales se planifican según se terminan las tareas anteriores. La duración media de estos ha sido una semana de trabajo.

En cada uno de los \emph{sprints} se generan una serie de tareas las cuales tienen que ser realizadas en ese intervalo de tiempo. Para ello se utiliza la plataforma \emph{Zenhub}, en la cual las \emph{issues} hacen de tareas, y los \emph{Milestones} de \emph{Sprints}.

\subsection{Sprint 0}

Este primer \emph{sprint} fue el más complicado de planificar. Estaba incluido el desarrollo del \emph{hardware} el cual no sabíamos seguro si se conseguirían los conocimientos necesarios para diseñarlo, ni si después funcionaría correctamente. Además, el desarrollo de la aplicación dependía del resultado de esta parte ya que iba a basarse en ello.

Al final, sobrepasando un poco el tiempo planificado, se consiguió poner el \emph{hardware} en marcha, con lo que se podía comenzar con el desarrollo del \emph{software}.


\subsection{Sprint 1}

Los objetivos de este \emph{sprint} fueron asentar la idea general de la aplicación a desarrollar, su estructura principal y funcionalidades.

Además se comenzó a diseñar la estructura de la base de datos y el acceso al \emph{hardware} a través de la comunicación con los puertos serie.

Las horas de trabajo fueron sobrepasadas sobre lo planificado en el \emph{Sprint}.

\imagen{sprint1}{\emph{Sprint} 1.}

\subsection{Sprint 2}

Los objetivos de este \emph{sprint} fueron comenzar con el desarrollo de la interfaz gráfica. Para ello era necesario la creación de gráficas de forma dinámica. Estas serían utilizadas para monitorizar los datos que fluían por el bus CAN en tiempo real.

Esta fase llevó un poco más tiempo de lo esperado.

\imagen{sprint2}{\emph{Sprint} 2.}

\subsection{Sprint 3}

En este \emph{sprint} se planteó la posibilidad de poder importar y exportar un proyecto de la aplicación. Para ello, se decidió usar ficheros JSON, los cuales contendrían todos los datos del proyecto.

En este punto se planteó la posibilidad de realizar una parte de la aplicación sobre una tecnología web. La idea era generar los ficheros JSON desde la aplicación Java para posteriormente poder importarlos a la web para su monitorización desde la misma.

\imagen{sprint3}{\emph{Sprint} 3.}

\subsection{Sprint 4}

El objetivo de este \emph{sprint} era la generación de un gestor de proyectos interno en la aplicación. Para ello era necesario diseñar las funcionalidades para la creación de un nuevo proyecto, la asignación de valores identificados al mismo, así como sus opciones de guardar y eliminar de la base de datos.

\imagen{sprint4}{\emph{Sprint} 4.}

\subsection{Sprint 5}

En este \emph{sprint} se desarrolló la interacción con la base de datos, tanto para obtener datos de ella como para recuperarlos. Además se realizó un tratado de las posibles excepciones que puedan surgir por problemas internos de la aplicación.

\imagen{sprint5}{\emph{Sprint} 5.}

\subsection{Sprint 6}

En este \emph{sprint} se centraron los esfuerzos en tener un borrador de la memoria y los anexos. Además, el desarrollo de la aplicación continuaba añadiendo opciones como la de ver los datos de una ID y un \emph{byte} concretos en tiempo real. Esta opción estaría disponible para todos los datos mostrados por pantalla.

\imagen{sprint6}{\emph{Sprint} 6.}

\subsection{Sprint 7}

En este último \emph{sprint} se trató de concluir con la documentación y terminar de pulir las pequeñas características que quedaban pendientes del proyecto. También se centraron esfuerzos en concluir la documentación y los vídeos de prueba de la aplicación, parte importante ya que sin el \emph{hardware} no es posible probar la aplicación.


\subsection{Resumen}

En la siguiente tabla se muestran los tiempos dedicados a cada uno de los distintos tipos de tareas:



\begin{longtable}[]{@{}lrr@{}}
\toprule
\begin{minipage}[b]{0.37\columnwidth}\raggedright\strut
Categoría\strut
\end{minipage} & \begin{minipage}[b]{0.37\columnwidth}\raggedleft\strut
Tiempo (horas)\strut
\end{minipage}\tabularnewline
\midrule
\endhead
\begin{minipage}[t]{0.37\columnwidth}\raggedright\strut
\emph{Documentación}\strut
\end{minipage} & \begin{minipage}[t]{0.37\columnwidth}\raggedleft\strut
67\strut
\end{minipage}\tabularnewline
\begin{minipage}[t]{0.37\columnwidth}\raggedright\strut
\emph{Características}\strut
\end{minipage} & \begin{minipage}[t]{0.37\columnwidth}\raggedleft\strut
132\strut
\end{minipage}\tabularnewline
\begin{minipage}[t]{0.37\columnwidth}\raggedright\strut
\emph{Investigación}\strut
\end{minipage}& \begin{minipage}[t]{0.37\columnwidth}\raggedleft\strut
42\strut
\end{minipage}\tabularnewline
\begin{minipage}[t]{0.37\columnwidth}\raggedright\strut
\emph{Corrección de errores}\strut
\end{minipage} & \begin{minipage}[t]{0.37\columnwidth}\raggedleft\strut
28\strut
\end{minipage}\tabularnewline
\midrule
\begin{minipage}[t]{0.37\columnwidth}\raggedright\strut
Total\strut
\end{minipage} & \begin{minipage}[t]{0.37\columnwidth}\raggedleft\strut
269\strut
\end{minipage}\tabularnewline
\bottomrule
\caption{Horas dedicadas al proyecto.}
\end{longtable}


\imagen{resumenHoras}{Resumen de horas dedicadas al proyecto.}


\section{Estudio de viabilidad}

En este punto se van a analizar por separado la viabilidad económica y la viabilidad legal del proyecto.

\subsection{Viabilidad económica}

\subsection{Costes}\label{costes}

Los costes van a ser desglosados en dos partes:

\textbf{Costes de personal:}

El proyecto ha sido desarrollado por una sola persona, empleada a tiempo completo durante tres meses. Considerando los siguientes valores:


\begin{longtable}[]{@{}lr@{}}
\toprule
\begin{minipage}[b]{0.38\columnwidth}\raggedright\strut
\textbf{Concepto}\strut
\end{minipage} & \begin{minipage}[b]{0.20\columnwidth}\raggedleft\strut
\textbf{Coste \euro{}}\strut
\end{minipage}\tabularnewline
\midrule
\endhead
\begin{minipage}[t]{0.38\columnwidth}\raggedright\strut
Salario mensual neto\strut
\end{minipage} & \begin{minipage}[t]{0.20\columnwidth}\raggedleft\strut
{1008,15}\strut
\end{minipage}\tabularnewline
\begin{minipage}[t]{0.38\columnwidth}\raggedright\strut
IRPF (9,64\%)\strut
\end{minipage} & \begin{minipage}[t]{0.20\columnwidth}\raggedleft\strut
115,68\strut
\end{minipage}\tabularnewline
\begin{minipage}[t]{0.38\columnwidth}\raggedright\strut
SS (30 \%)\strut
\end{minipage} & \begin{minipage}[t]{0.20\columnwidth}\raggedleft\strut
360,00\strut
\end{minipage}\tabularnewline
\begin{minipage}[t]{0.38\columnwidth}\raggedright\strut
Salario mensual bruto\strut
\end{minipage} & \begin{minipage}[t]{0.20\columnwidth}\raggedleft\strut
1200,00\strut
\end{minipage}\tabularnewline
\midrule
\begin{minipage}[t]{0.38\columnwidth}\raggedright\strut
\textbf{Total tres meses}\strut
\end{minipage} & \begin{minipage}[t]{0.20\columnwidth}\raggedleft\strut
3600\strut
\end{minipage}\tabularnewline
\bottomrule
\caption{Costes de personal.}
\end{longtable}

\textbf{Costes de \emph{hardware}:}

En este apartado se revisan los costes del \emph{hardware} desarrollado en el proyecto.

\begin{longtable}[]{@{}lr@{}}
\toprule
\begin{minipage}[b]{0.38\columnwidth}\raggedright\strut
\textbf{Concepto}\strut
\end{minipage} & \begin{minipage}[b]{0.20\columnwidth}\raggedleft\strut
\textbf{Coste \euro{}}\strut
\end{minipage}\tabularnewline
\midrule
\endhead
\begin{minipage}[t]{0.38\columnwidth}\raggedright\strut
PCB (10 prototipos)\strut
\end{minipage} & \begin{minipage}[t]{0.20\columnwidth}\raggedleft\strut
1,75\strut
\end{minipage}\tabularnewline
\begin{minipage}[t]{0.38\columnwidth}\raggedright\strut
Transporte\strut
\end{minipage} & \begin{minipage}[t]{0.20\columnwidth}\raggedleft\strut
7,11\strut
\end{minipage}\tabularnewline
\begin{minipage}[t]{0.38\columnwidth}\raggedright\strut
Componentes (para 2 uds)\strut
\end{minipage} & \begin{minipage}[t]{0.20\columnwidth}\raggedleft\strut
51,90\strut
\end{minipage}\tabularnewline
\begin{minipage}[t]{0.38\columnwidth}\raggedright\strut
Aduanas\strut
\end{minipage} & \begin{minipage}[t]{0.20\columnwidth}\raggedleft\strut
15,00\strut
\end{minipage}\tabularnewline
\begin{minipage}[t]{0.38\columnwidth}\raggedright\strut
Soldador\strut
\end{minipage} & \begin{minipage}[t]{0.20\columnwidth}\raggedleft\strut
15,00\strut
\end{minipage}\tabularnewline
\begin{minipage}[t]{0.38\columnwidth}\raggedright\strut
Estaño\strut
\end{minipage} & \begin{minipage}[t]{0.20\columnwidth}\raggedleft\strut
5,00\strut
\end{minipage}\tabularnewline
\begin{minipage}[t]{0.38\columnwidth}\raggedright\strut
Flux\strut
\end{minipage} & \begin{minipage}[t]{0.20\columnwidth}\raggedleft\strut
3,00\strut
\end{minipage}\tabularnewline
\begin{minipage}[t]{0.38\columnwidth}\raggedright\strut
Ordenador\strut
\end{minipage} & \begin{minipage}[t]{0.20\columnwidth}\raggedleft\strut
855,47\strut
\end{minipage}\tabularnewline
\midrule
\begin{minipage}[t]{0.38\columnwidth}\raggedright\strut
\textbf{Total}\strut
\end{minipage} & \begin{minipage}[t]{0.20\columnwidth}\raggedleft\strut
954,23\strut
\end{minipage}\tabularnewline
\bottomrule
\caption{Costes de personal.}
\end{longtable}



\textbf{Costes totales:}

\begin{longtable}[]{@{}lr@{}}
\toprule
\begin{minipage}[b]{0.38\columnwidth}\raggedright\strut
\textbf{Concepto}\strut
\end{minipage} & \begin{minipage}[b]{0.20\columnwidth}\raggedleft\strut
\textbf{Coste \euro{}}\strut
\end{minipage}\tabularnewline
\midrule
\endhead
\begin{minipage}[t]{0.38\columnwidth}\raggedright\strut
Personal\strut
\end{minipage} & \begin{minipage}[t]{0.20\columnwidth}\raggedleft\strut
3600 \euro{}\strut
\end{minipage}\tabularnewline
\begin{minipage}[t]{0.38\columnwidth}\raggedright\strut
\emph{Hardware}\strut
\end{minipage} & \begin{minipage}[t]{0.20\columnwidth}\raggedleft\strut
954,23 \euro{}\strut
\end{minipage}\tabularnewline
\midrule
\begin{minipage}[t]{0.38\columnwidth}\raggedright\strut
\textbf{Total}\strut
\end{minipage} & \begin{minipage}[t]{0.20\columnwidth}\raggedleft\strut
4554,23 \euro{}\strut
\end{minipage}\tabularnewline
\bottomrule
\caption{Costes totales.}
\end{longtable}

\textit{Nota: No se ha tenido en cuenta en el cálculo anterior el coste de los tutores.}

Teniendo en cuenta los datos anteriores, para hacer rentable el proyecto con un periodo de tiempo de un año, sería necesario vender tanto el \emph{hardware} como licencias de la aplicación. 

\subsection{Viabilidad legal}

A continuación se exponen todos los temas relacionados con las licencias del software y \emph{hardware} de proyecto, así como de la documentación e imágenes.

Tabla de licencias de las dependencias utilizadas:

\begin{longtable}[]{@{}llll@{}}
\toprule
\begin{minipage}[b]{0.18\columnwidth}\raggedright\strut
Dependencia\strut
\end{minipage} & \begin{minipage}[b]{0.10\columnwidth}\raggedright\strut
Versión\strut
\end{minipage} & \begin{minipage}[b]{0.49\columnwidth}\raggedright\strut
Descripción\strut
\end{minipage} & \begin{minipage}[b]{0.11\columnwidth}\raggedright\strut
Licencia\strut
\end{minipage}\tabularnewline
\midrule
\endhead
\begin{minipage}[t]{0.18\columnwidth}\raggedright\strut
Java\strut
\end{minipage} & \begin{minipage}[t]{0.08\columnwidth}\raggedright\strut
8.0.191\strut
\end{minipage} & \begin{minipage}[t]{0.49\columnwidth}\raggedright\strut
JDK\strut
\end{minipage} & \begin{minipage}[t]{0.11\columnwidth}\raggedright\strut
GPL\strut
\end{minipage}\tabularnewline
\begin{minipage}[t]{0.18\columnwidth}\raggedright\strut
USBtinLib\strut
\end{minipage} & \begin{minipage}[t]{0.08\columnwidth}\raggedright\strut
1.2.0\strut
\end{minipage} & \begin{minipage}[t]{0.49\columnwidth}\raggedright\strut
Librería para la conexión con el \emph{hardware}.\strut
\end{minipage} & \begin{minipage}[t]{0.11\columnwidth}\raggedright\strut
GPL\strut
\end{minipage}\tabularnewline
\begin{minipage}[t]{0.18\columnwidth}\raggedright\strut
JSSC\strut
\end{minipage} & \begin{minipage}[t]{0.08\columnwidth}\raggedright\strut
2.8.0\strut
\end{minipage} & \begin{minipage}[t]{0.49\columnwidth}\raggedright\strut
Librería para la búsqueda de puertos serie.\strut
\end{minipage} & \begin{minipage}[t]{0.11\columnwidth}\raggedright\strut
LGPL\strut
\end{minipage}\tabularnewline
\begin{minipage}[t]{0.18\columnwidth}\raggedright\strut
sqlite-jdbc\strut
\end{minipage} & \begin{minipage}[t]{0.08\columnwidth}\raggedright\strut
3.23.1\strut
\end{minipage} & \begin{minipage}[t]{0.49\columnwidth}\raggedright\strut
Librería para realizar la conexión con la base de datos.\strut
\end{minipage} & \begin{minipage}[t]{0.11\columnwidth}\raggedright\strut
Apache 2.0\strut
\end{minipage}\tabularnewline
\begin{minipage}[t]{0.18\columnwidth}\raggedright\strut
json\strut
\end{minipage} & \begin{minipage}[t]{0.08\columnwidth}\raggedright\strut
1.6\strut
\end{minipage} & \begin{minipage}[t]{0.49\columnwidth}\raggedright\strut
Librería para la importación y exportación de ficheros JSON.\strut
\end{minipage} & \begin{minipage}[t]{0.11\columnwidth}\raggedright\strut
JSON\strut
\end{minipage}\tabularnewline
\bottomrule
\caption{Dependencias del proyecto y sus licencias}
\end{longtable}


Teniendo en cuenta todas las licencias de las dependencias utilizadas, la licencia que más se ajusta a nuestro proyecto es la \emph{GNU General Public License v3.0} más conocida como \emph{GNU GPL}.

Con el uso de esta licencia, nos exponemos a las siguientes condiciones:


\begin{longtable}[]{@{}lll@{}}
\toprule
\begin{minipage}[b]{0.20\columnwidth}\raggedright\strut
Derechos\strut
\end{minipage} & \begin{minipage}[b]{0.33\columnwidth}\raggedright\strut
Condiciones\strut
\end{minipage} & \begin{minipage}[b]{0.33\columnwidth}\raggedright\strut
Limitaciones\strut
\end{minipage}\tabularnewline
\midrule
\endhead
\begin{minipage}[t]{0.20\columnwidth}\raggedright\strut
Uso comercial\strut
\end{minipage} & \begin{minipage}[t]{0.33\columnwidth}\raggedright\strut
liberar código fuente.\strut
\end{minipage} & \begin{minipage}[t]{0.33\columnwidth}\raggedright\strut
Limitación de responsabilidad.\strut
\end{minipage}\tabularnewline
\begin{minipage}[t]{0.20\columnwidth}\raggedright\strut
Modificación\strut
\end{minipage} & \begin{minipage}[t]{0.33\columnwidth}\raggedright\strut
Nota sobre la licencia y copyright.\strut
\end{minipage} & \begin{minipage}[t]{0.33\columnwidth}\raggedright\strut
Sin garantías.\strut
\end{minipage}\tabularnewline
\begin{minipage}[t]{0.20\columnwidth}\raggedright\strut
Distribución\strut
\end{minipage} & \begin{minipage}[t]{0.33\columnwidth}\raggedright\strut
Modificaciones bajo la misma licencia.\strut
\end{minipage} & \begin{minipage}[t]{0.33\columnwidth}\raggedright\strut
\strut
\end{minipage}\tabularnewline
\bottomrule
\caption{Derechos y condiciones de la licencia GNU GPL.}
\end{longtable}


Respecto al \emph{hardware} el cual está basado en el proyecto \emph{USBtin}, no se especifica una licencia por parte del autor del mismo.

Por otra parte, el \emph{software} utilizado para el diseño de la caja y para el desarrollo del \emph{hardware} son completamente gratuitos.
