\capitulo{5}{Aspectos relevantes del desarrollo del proyecto}

A continuación se desarrollan los aspectos más importantes que influyeron en el desarrollo del proyecto.

\section{Inicio del proyecto}\label{inicio_del_proyecto}

La idea de desarrollar esta aplicación surgió unos meses atrás, tras trabajar en una empresa que de dedicaba a la reparación de maquinaria.

Tras investigar y leer algunos manuales de reparación de esas máquinas, era de destacar que existía una característica en común, que era el uso del bus CAN en muchas de ellas, por no decir todas. A partir de ahí surgió el interés y las ganas de indagar sobre su funcionamiento.

Una vez fue posible la lectura de dichos datos, quedaba clara la dificultad para la identificación de los mismos, ya que todos ellos estaban en hexadecimal, lo cual dificultaba su identificación.

Tras el desarrollo de un pequeños \emph{script} en python los cuales eran capaces de realizar una gráfica con los datos previamente adquiridos del bus CAN y detectar la mayor facilidad a la hora de identificar esos datos, surgió la idea de realizar este proyecto.

Existía la necesidad de que, al contrario que en el caso anterior, este proyecto permitiera realizar ese proceso en tiempo real, no con una muestra de datos tomada anteriormente.

Otra de las principales motivaciones fue el hecho de que si por ejemplo, existiera la necesidad de recoger los datos de un sensor (de velocidad de una rueda) para su monitorización y este dato es transmitido a través del bus CAN, no es necesaria la instalación de un nuevo sensor, abaratando costes y facilitando realizar la tarea necesaria.

\section{Metodologías}\label{metodologias}

Se decidió utilizar la metodología \emph{Scrum}, una metodología ágil, de forma que no era necesario definir el proyecto desde el principio (no se conocía el alcance del mismo, ni si iba a ser posible realizar ciertas propuestas planteadas).

La duración aproximada de los \emph{sprints} era de una semana, a la vez que al finalizar dicho \emph{sprint}, se planificaba el de la semana siguiente.

A través del uso de la herramienta \emph{ZenHub}, se generaban \emph{issues}, es decir, tareas a realizar en ese \emph{sprint}, y se cambiaban de estado según se iban finalizando.

\section{Formación}\label{formacion}

El proyecto requería conceptos tanto de funcionamiento como de la implementación del bus CAN, los cuales desconocía en un principio. A través de los siguientes documentos, se adquirieron conocimientos sobre la materia:


LINKS CAN BUS


\section{Desarrollo del \emph{hardware}}\label{desarrollo_del_hardware}

Se decidió comenzar por esta fase por los siguientes motivos:

\begin{itemize}
\item
No era seguro que el \emph{hardware} fuera a funcionar con lo que no se podía crear una dependencia por parte del software en caso de que esa parte no se realizase satisfactoriamente.

\item
La producción de la PCB requiere de unas semanas desde que envías el pedido hasta que el material llega.
\item
Era la parte más desconocida y en la cual habría que invertir más tiempo para aplicar la metodología sugerida a la hora del desarrollo de este tipo de proyectos.
\end{itemize}

Se podrían definir 4 etapas las cuales se han llevado a cabo para el desarrollo:

\subsection{Diseño del esquema eléctrico}\label{diseño_del_esquema_electrico}

El primer paso para el desarrollo de una placa de circuito impreso, es diseñar el esquema en el que se incluyen los componentes, y la conexión que se va a realizar entre ellos. En este caso, el objetivo era duplicar un esquema existente, introduciéndole un \emph{Hub USB}, todo ello en la misma placa.

En este esquema se especifican los componentes utilizados (resistencias, condensadores, circuitos integrados, etc). Además se define que entradas y salidas de esos componentes van conectados al resto de componentes.

\imagen{scheme}{Vista general del esquema.}

\subsection{Generación del \emph{netlist}}\

Una vez realizado el esquema eléctrico, necesitamos generar el fichero \emph{Netlist}. En este fichero se almacenan los siguientes datos:

Por cada componente:

\begin{itemize}
\item
Se describen las conexiones de cada uno de los elementos del circuito.
\item
El tipo de elemento que es cada componente.
\item
La huella (\emph{footprint}) que va a utilizar, es decir, en función de cada encapsulado (forma física del \emph{chip}), será necesario que la zona de soldadura de la PCB sea adaptada al mismo, tanto en tamaño como en forma.
\end{itemize}


\subsection{Posicionamiento y enrutado de los componentes y las pistas en la PCB}\label{posicionamiento_y_enrutado_de_los_componentes_y_las_pistas_en_la_pcb}

Una vez se ha generado el fichero \emph{Netlist}, este es cargado en el editor de PCBs. Para que nos hagamos una idea, ese es el resultado:


En este momento, es cuando tenemos que decidir:

\begin{itemize}
\item
El número de capas que va a tener nuestra PCB: En este caso se utilizaron 2 capas al tratarse de un proyecto sencillo, pero en el que necesitaban hacerse cruces de pistas.
\item
La posición de los componentes y el enrutado de las pistas.
\end{itemize}

FOTO DEL ESQUEMA TERMINADO.


\imagen{fotoPCB}{PCB del proyecto sin los componentes}


Tras limpiar la PCB, posicionar y soldar los componentes, obtenemos el siguiente resultado:


\imagen{finishedPCB}{PCB del proyecto con los componentes soldados}

\section{Desarrollo del \emph{software}}

Lo primero que había que decidir era el lenguaje en el que se iba a desarrollar la aplicación. Se barajó la posibilidad de realizarlo en Java o Python. La decisión de Java se debió a la disposición de una librería compatible con el \emph{hardware} y a la librería JavaFX la cual nos permite realizar una aplicación muy visual y amigable de forma sencilla y eficiente.

Una vez elegido el lenguaje de desarrollo, se comenzó con la modelación de la interfaz gráfica, es decir, de la estructura de la aplicación a nivel visual. Una vez estaba claro, se comenzó con el desarrollo del código.

Para la base de datos, se decidió utilizar SQLite, ya que no requiere de ningún motor de base de datos, sino simplemente la utilización de una librería interna en la aplicación. Además, la información almacenada iba a ser muy pequeña y era más que suficiente.

Principalmente surgieron dos grandes problemas durante el desarrollo de la parte del software, ambos relacionados con la parte de la interfaz gráfica.

\subsection{Generación de gráficas de forma dinámica}\label{generacion_de_graficas_de_forma_dinamica}

Era necesario que las gráficas se fueran creando de manera dinámica cuando existiera un nuevo dato a representar, no tener una estructura estática creada previamente. Para ello, cada vez que un nuevo \emph{Frame} del bus CAN fuera detectado, había que proceder a la creación de dichos elementos, así como de un \emph{listener} en ese objeto, el cual consultase la estructura en la que son almacenados los datos, para su posterior representación gráfica.

Para ello se decidió utilizar una estructura gráfica de tipo \emph{GridPane}, dividiendo la pantalla en el eje horizontal en 3 celdas. Cada una de estas celdas se iría rellenando en función de las necesidades e iría incrementandose de arriba a abajo si fuera necesario. Para conseguir ese efecto, el elemento padre del \emph{GridPane} tenía que ser un \emph{ScrollPane} es decir, lo que conocemos como un \emph{Scroll}.

Todo esto tenía que ir de la mano con una serie de estructuras, también generadas dinámicamente, para realizar la lectura y escritura de los datos que tienen que ser representados.

De esa manera, se decidió utilizar estructuras de tipo \emph{HashMap}, las cuales identificaban el dato por el par \emph{ID} y \emph{número de byte representado}.

\subsection{Problemas de concurrencia}\label{problemas_de_concurrencia}

Los elementos que representan gráficamente los datos (LineChart), realizan una lectura de un tipo de datos llamado \emph{ObservableList}. Durante la ejecución de la aplicación existen 2 \emph{threads}, uno para la ejecución de la parte gráfica, y el otro para la lectura de los datos del bus CAN.

El problema surge cuando se intenta realizar una escritura en el objeto de tipo \emph{ObservableList} desde el \emph{thread} que lee el bus CAN al mismo tiempo que el \emph{thread} de la interfaz gráfica realiza una lectura de los datos del mismo objeto.

Para la resolución de este problema se decidió utilizar un modelo de tipo Productor/Consumidor. Para ello se definió un objeto de tipo \emph{ConcurrentLinkedQueue} en el cual el \emph{thread} encargado de la lectura del bus CAN, iba introduciendo la información que le llegaba al tiempo que el \emph{thread} de la interfaz gráfica iba cargando esos datos en el objeto de tipo \emph{ObservableList}. De esta manera, se consiguieron solucionar los problemas de concurrencia.

\section{Diseño de la interfaz gráfica}\label{diseño_de_la_interfaz_grafica}

Para el desarrollo de la interfaz gráfica, se ha utilizado la herramienta \emph{SceneBuilder}.

Esta herramienta nos permite crear interfaces gráficas de forma fácil e intuitiva. Una vez realizado el diseño, la herramienta genera un fichero FXML (basado en una estructura XML), el cual será utilizado para mostrar los elementos en pantalla.

En este fichero es importante tener en cuenta los siguientes aspectos:

\begin{itemize}

\item
Debemos de indicarle que controlador se va a encargar de manejar esa vista. Es decir, siempre tiene que existir una relación vista-controlador.
\item
Todos los componentes que vayan a ser utilizados desde el controlador, tiene que tener como mínimo la descripción \emph{fx:id} para su identificación. Esta id tiene que ser única en el documento FXML.
\item
Además, en este fichero se indican las acciones que queremos que tengan los elementos. Por ejemplo, que al pulsar un botón o modificar un texto, el controlador ejecute una función establecida.

\end{itemize}



\section{Testing}\label{testing}

Para realizar las pruebas correspondientes durante el desarrollo de la aplicación y no depender del acceso físico a una máquina o vehículo, se ha desarrollado una pequeña placa la cual simularía una red CAN, en la que se pueden conectar dos nodos, uno para enviar datos al bus y el otro para recibir.

Para ello simplemente se han colocado dos resistencias de 120 \emph{ohmns} (definidas por el estándar del CAN), y unos de tipo DB9, ya que son los más habituales en este tipo de \emph{hardware}.

\imagen{hardwareRedCan}{Placa casera para simular una red CAN}.

\section{Diseño e impresión 3D de una caja protectora}\label{desarrollo_e_impresion_3d_de_una_caja_protectora}

Para la protección de \emph{hardware} desarrollado, se vio la necesidad de crear una caja la cual almacenase dentro la placa, pero que además permitiera la conexión de los elementos necesarios para su funcionamiento.

Para ello se ha utilizado el \emph{software} \emph{SolidWorks} para el diseño de la misma.

FOTO CAJA EN 3D EN EL PC

Una vez concluido el diseño, el fichero 3D ha sido introducido en \emph{Ultimaker CURA}, un \emph{silicer} el cual nos transforma el objeto 3D en un fichero el cual la impresora 3D sea capaz de interpretar.

Para realizar la impresión, se ha utilizado una impresora 3D \emph{BQ Prusa i3 Hephestos} y plástico de tipo ABS (\emph{Acrylonitrile Butadiene Styrene}) el cual resulta más difícil de imprimir, pero es adecuado para un uso intensivo como puede ser el de la industria, teniendo más resistencia a golpes y a temperatura que uno de sus competidores como puede ser el PLA ( Polylactic acid).


FOTO DEL A CAJA TERMINADA

