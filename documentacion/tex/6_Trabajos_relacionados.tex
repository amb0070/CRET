\capitulo{6}{Trabajos relacionados}

Cada fabricante dispone de sus propias herramientas (al igual que muchos de ellos disponen de su propio hardware). Estas herramientas, como se ha mencionado en la introducción, tienen un coste muy elevado en la mayoría de los casos.


Las principales ventajas de este proyecto sobre los otros mencionados anteriormente son:

El hardware desarrollado puede ser producido por cualquier persona, ya que los esquemas y los ficheros necesarios para su producción son open source.

Aplicación multiplataforma, compatible tanto con sistemas Windows como Linux.

No es necesaria la instalación de drivers en el equipo en el que se va a utilizar la herramienta.

Es la única herramienta libre que permite "graficar" los datos que fluyen por el bus CAN, así como su identificación y posterior monitorización.


Las principales desventajas son:

Actualmente solo funciona con un hardware en concreto. Como se menciona en la sección --Lineas futuras de trabajo--, el desarrollo de este proyecto continuaría con la incorporación de un módulo para el soporte de los drivers SocketCAN en sistemas linux.