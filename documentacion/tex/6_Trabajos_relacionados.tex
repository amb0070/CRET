\capitulo{6}{Trabajos relacionados}


Cada fabricante dispone de sus propias herramientas (al igual que muchos de ellos disponen de su propio \emph{hardware}). Estas herramientas, como se ha mencionado en la introducción, tienen un coste muy elevado en la mayoría de los casos.

Las principales ventajas de este proyecto son:

\begin{itemize}

\item
El \emph{hardware} desarrollado puede ser producido por cualquier persona, ya que los esquemas y los ficheros necesarios para su producción son \emph{open source}.
\item
Aplicación multiplataforma, compatible tanto con sistemas Windows como Linux.
\item
No es necesaria la instalación de \emph{drivers} en el equipo en el que se va a utilizar la herramienta.
\item
Es la única herramienta libre que permite graficar los datos que fluyen por el bus CAN, así como su identificación y posterior monitorización.

\end{itemize}

Las principales desventajas son:

\begin{itemize}
\item
Actualmente solo funciona con un \emph{hardware} en concreto. Como se menciona en la sección Líneas futuras de trabajo \ref{lineas_de_trabajo_futuras}, el desarrollo de este proyecto continuaría con la incorporación de un módulo para el soporte de los \emph{drivers} \emph{SocketCAN} en sistemas \emph{Linux}.
\end{itemize}
\newpage
A continuación se realiza una pequeña comparación entre las herramientas:

\tablaSmall{Comparación entre CRET y otras herramientas del mercado}{p{3cm} c c c c}{herramientasportipodeuso}
{ \multicolumn{1}{l}{Herramientas} & CRET & CANalyzat0r & PCAN Explorer \\}{
Visualización de los datos en RAW & X & X &X\\
Multiplataforma & X & X &\\
Hardware libre & X & N/A &\\
Representación gráfica de los datos & X & & X\\
Etiquetado con un \emph{click} & X & &\\
Importar y exportar proyectos JSON & X & &\\
} 


Cabe destacar la presencia del proyecto \emph{CANalyzat0r}\cite{ canalyzator:can}, el cual nos permite también el análisis de los datos que pasan por el bus CAN, pero que no nos permite una rápida identificación a través de la visualización de los datos. 

Además a través de CRET, se facilita mucho la identificación de las señales, pudiendo etiquetarlas y almacenarlas en el \emph{dashboard}, y a su vez en una base de datos, para su utilización más adelante. En \emph{CANalyzat0r}, el etiquetado de los datos identificados tiene que hacerse introduciendo la ID específica, además de que no nos permite separar los \emph{bytes} del campo de datos.
